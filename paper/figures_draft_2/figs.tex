\documentclass[10pt,letterpaper]{article}
\usepackage[top=0.85in,left=2.75in,footskip=0.75in]{geometry}
\usepackage{bibentry}

% Text layout specific to Supplemental Materials
\topmargin 0.0cm
\oddsidemargin 0.5cm
\evensidemargin 0.5cm
\textwidth 16cm
\textheight 21cm

\setlength{\parskip}{1em}

\input{header.tex}
\pagestyle{empty} %%in order to delete the number at the bottom of the page


\begin{document}

\section*{Figures}
\newpage

\begin{figure*}[h!]
\centering
\includegraphics[height=6in, width=7in]{figures2.split/figures2_1.pdf}
\caption{\textbf{EDLogo representation and its performance comparison to standard logo plots}
      We present a comparative study of the \textit{EDLogo} representation with respect to the standard logo plots, through three examples. In the first example (panel (A)), we compare the \textit{EDLogo} representation (\textit{right}) with the standard information content based logo plot  (\textit{left}) for modeling the transcription factor binding site of EBF1-disc1 transcription factor. We observe that the \textit{EDLogo} plot identifies the depletion in the middle of the sequence much better than the standard plot. In panel (B), we compare the \textit{EDLogo} plot with the position speciifc scoring matrix (PSSM) plot of the binding sequence of the Bacterial transcription activator, effector binding domain protein PF06445 (motif 4, Start=153 Length=8). We observe that the \textit{EDLogo} representation is much more visually parsimonious and detailed than the PSSM plot. In panel (C), we present the standard logo and the \text{EDLogo} representation of the mutational signature profile of the all mutations in lymphoma B cells, where the data is due to Alexandrov et al 2013 \cite{Alexandrov2013}. We observe that the depletion of G to the right of the mutation - possibly occurring due to methylated CpG sites being less prone to mutation - much more clearly in the \textit{EDLogo} representation compared to the standard logo plot.}
\label{fig:fig1}
\end{figure*}

\end{document}

